
%%% This is an example file for the Auburn University style options
%%%       aums.sty (Masters Thesis)
%%%       auphd.sty (Ph.D. Dissertation)
%%%       auhonors.sty (Honors Scholar)

%%%To use it, please edit the necessary options, title, author, date, year, keywords, advisor, professor, etc. 

\documentclass[12pt]{report}
\usepackage{auphd}     % For Ph.D.
\usepackage{ulem}       % underlining on style-page; see 
\usepackage{url}
\usepackage{tikz}
\usepackage{pgf}
\usepackage{subfiles}
\usepackage{blindtext}
\usepackage{hhline}
\usepackage[intoc]{nomencl}
\usepackage{hyperref}
\usepackage[english]{babel}
\usepackage{listings}
\usepackage{color}
\usepackage[margin=1in]{geometry}
\usepackage[toc,page]{appendix}
\usepackage{titlesec}
\usepackage{lipsum}
\usepackage[tracking=true]{microtype}

\graphicspath{ {images/} }

\renewcommand{\nomname}{List of Abbreviations}   	       
\makenomenclature 




%%%%%Format rules: Normal margins are 1 in. If you need to print with 1.5in margins, uncomment the line below
%\oddsidemargin0.5in \textwidth6in

%% If you do not need a List of Abbreviations, then comment out the lines below and the \printnomenclature line.
%%for List of Abbreviations information:  (see http://www.mackichan.com/TECHTALK/509.htm  )

%% don't forget to run:   makeindex ausample.nlo -s nomencl.ist -o ausample.nls
%% Also, if 




% May want theorems numbered by chapter
\newtheorem{theorem}{Theorem}[chapter]

% Put the title, author, and date in. 
\title{M-TRA: A Multi-Tiered Resilient Architecture for Cyber-Physical Systems}
\author{Janice Ca\~nedo} 
\date{December 2017} %date of graduation
\copyrightyear{2017} %copyright year

\keywords{Internet of Things, Architecture, Security, Cyber-Physical Systems}

% Put the Thesis Adviser here. 
\adviser{Anthony Skjellum}


% Put the committee here (including the adviser), one \professor for each. 
% The advisor must be first, and the dean of the graduate school must be last.
\professor{Anthony Skjellum, Professor of Computer Science and Software Engineering}
%\professor{David Umphress, Professor of Computer Science and Software Engineering}



\begin{document}

\begin{romanpages}      % roman-numbered pages 

\TitlePage 

%\begin{abstract} 
%Place the text of the abstract here. Headings come automatically.
%\end{abstract}

\subfile{chapters/abstract.tex}{}

%\begin{acknowledgments}
%Put text of the acknowledgments here.
%\end{acknowledgments}

\tableofcontents
\listoffigures
\listoftables

\printnomenclature[0.5in] %used for the List of Abbreviations
\end{romanpages}        % All done with roman-numbered pages


\normalem       % Make italics the default for \em


\normalem       % Make italics the default for \em

\chapter{Overview}  % Use \\ for long titles 

\subfile{chapters/overview.tex}{}

\section{Summary}

IoT/CPS is quickly evolving and becoming integrated more in our everyday world. These systems require the ability to scale from small home automation systems to large-scale smart grids. With CPS, it is necessary that these devices and systems are secure and resilient. There is a need to investigate means to provide security within these systems including lightweight device handshakes and active monitoring for anomalies and intrusions. Along with security, resiliency within a IoT/CPS system is of concern. Resiliency requires redundancy of communication and ability to recover from an anomaly. Finally, the ability to maintain a resilient and secure system must remain be scalable as a system grows from 10 devices to 1000s of devices. The goal of this dissertation will be to address these issues and outline a scalable, resilient and secure architecture for IoT/CPS Systems.

The remainder of this dissertation proposal is organized as follows: Chapter~\ref{chap:lit} is a review of literature covering CPS, security, scalability, and resiliency in IoT/CPS systems. Chapter~\ref{chap:questions} is an overview of research questions we intend to answer during this research. Finally, Chapter~\ref{chap:app} explains the approach and metrics we will use during the research process. 


\chapter{Review of Literature}
\label{chap:lit}

\subfile{chapters/related_work.tex}

\chapter{Research Questions}
\label{chap:questions}
\subfile{chapters/research_questions}{}

\chapter{Approach and Metrics}
\label{chap:app}
\subfile{chapters/approach.tex}{}

\subfile{chapters/metrics.tex}{}


{%\small
\bibliographystyle{unsrt}
\bibliographystyle{abbrv}
\bibliography{chapters/references}
}

\appendix
\chapter*{Appendices\addcontentsline{toc}{chapter}{Appendices}}


\begin{singlespace}

\chapter*{Appendices\addcontentsline{toc}{chapter}{Appendices}}
\subfile{chapters/publications-plans.tex}{}
\subfile{chapters/research_questions-scoping.tex}{}
\end{singlespace}



\end{document}

