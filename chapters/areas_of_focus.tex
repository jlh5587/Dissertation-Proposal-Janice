\documentclass[../main.tex]{subfiles}




\begin{document}

\section{System Architecture Attributes}

There are three key attributes within a cohesive IoT Architectures for a robust system including security, scalability, and resiliency. Below we define and identify key points of each area. 

\subsection{Secure Architecture}
Within an IoT system, there are multiple levels of security including network security, secure device handshakes, and protection against intrusion into the devices. According to H\"oller, the Security Model for IoT consists of communication security that focuses mostly on the confidentiality and integrity protection of interacting entities, and functional components such as Identity Management, Authentication, Authorization, and Trust \& Reputation \cite{m2m_iot}.

The SANS Institute defines network security as “the process of taking physical and software preventative measures to protect the underlying networking infrastructure from unauthorized access, misuse, malfunction, modification, destruction, or improper disclosure, thereby creating a secure platform for computers, users and programs to perform their permitted critical functions within a secure environment." \cite{network_sans}

First, verifying the validity of a device before it is allowed to enter the system is fundamental for a trusted system. Secure device handshakes can provide this security to a system. For an IoT system, the handshake needs to be light-weight enough to run on end devices, yet robust enough to not be easily replicated or faked. Second, monitoring of each device is important to enable verification that a device has not become corrupt.

Along with security within the IoT system, if an intrusion or defect is detected, it is necessary to have the ability to determine who or what caused the failure. DF can be implemented to provide a means to find this information. 

\subsection{Scalable Architecture}

When identifying scaling within an individual computer, there are two types of scaling: strong scaling and weak scaling. Strong scaling is CPU bound scaling. In strong scaling the problem size stays fixed while the number of processing elements are increased \cite{scaling}. Weak scaling is memory bound scaling. In this case the problem size (workload) assigned to each processing element stays constant and additional elements are used to solve a larger total problem (one that wouldn't fit in RAM on a single node, for example) \cite{scaling}. Scalability in parallel computing systems is the measure of its capacity to increase speedup in proportion to the increase in the number of processing elements \cite{parallel_computing}. 

Determining the speedup and efficiency within the systems as scaling occurs can be done by using Amdahl's law and Gustafson's law. Amdahl's law analyzes the speedup of a program given the addition of parallelization \cite{dr-dobbs}. Gustafson's law analyzes the increase in workload that can occur during a given time with the addition of parallelization. These two laws provide a foundation for determining the efficiency within the program is maximizing the use of the system. 


In IoT architecture, scalability includes scalability of operations within a gateway and scalability of increasing the number of devices included within the system. By analyzing weak and strong scaling within the gateway, we can investigate the ability to scale up the number of devices by using parallel processing with the CPU and GP-GPU. 

\subsection{Resilient Architecture}

According to Delic, Resiliency in an IoT system is defined in three parts \cite{Delic:2016:RIS:2891279.2822885}. First, an IoT system needs the ability to resist external perturbances and internal failures. Second, an IoT system should be capable of recovering and reentering a stable state after a failure occurs. Third, an IoT system should adapt its structure and behavior to constant change. 

%\subsection{Summary}

%This dissertation focuses on the security, scalablity, and resiliency within IoT/CPS systems. While each area can be analyzed independently, changes in each area can impact another. Our goal is to analyze systems of 10, 100, 1,000, and 10,000 devices to determine the impact of security, scalablity, resiliency as the number of devices increase.

\end{document}