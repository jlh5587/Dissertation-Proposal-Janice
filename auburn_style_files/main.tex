\documentclass[]{report}
\usepackage[utf8]{inputenc}
%\usepackage{auphd} 
\usepackage{ulem}       % underlining on style-page; see \normalem below
\usepackage{url}
\usepackage{tikz}
\usepackage{pgf}
\usepackage{url}
\usepackage{hyperref}
\usepackage[english]{babel}
\usepackage{listings}
\usepackage{color}
\usepackage[margin=1in]{geometry}
\usepackage[toc,page]{appendix}
\usepackage{titlesec}
\usepackage{lipsum}
\usepackage[tracking=true]{microtype}
\usepackage{subfiles}
\usepackage{blindtext}
\usepackage{hhline}
\usepackage[intoc]{nomencl}
\usepackage{setspace,lipsum}
\renewcommand{\nomname}{List of Abbreviations}   	       
\makenomenclature 

\graphicspath{ {images/} }



\titleformat{\chapter}[display]{\Large\bfseries}{Chapter \thechapter}{2pt}{\Large}


% May want theorems numbered by chapter
\newtheorem{theorem}{Theorem}[chapter]

% Put the title, author, and date in. 
\title{M-TRA: A Multi-Tiered Resilient Architecture for Cyber-Physical Systems}
\author{Janice Ca\~nedo} 
\date{October 2016} %date of graduation



\begin{document}
\pagenumbering{Roman}
\maketitle 

\subfile{chapters/abstract.tex}{}

\tableofcontents
%\listoffigures
%\listoftables

%\printnomenclature[0.5in] %used for the List of Abbreviations
     % All done with roman-numbered pages

\normalem       % Make italics the default for \em

\chapter{Overview}  % Use \\ for long titles 

\pagenumbering{arabic}
\subfile{chapters/overview.tex}{}

%\subfile{chapters/areas_of_focus.tex}{}

%\subfile{chapters/technology-overview.tex}{}

%\subfile{chapters/challenge_opportunity.tex}{}

%\subfile{chapters/statement_of_contribution.tex}{}

%\subfile{chapters/innovative_claims.tex}{}

%\subfile{chapters/broader_impact.tex}{}


\section{Summary}

IoT/CPS is quickly evolving and becoming integrated more in our everyday world. These systems require the ability to scale from small home automation systems to large-scale smart grids. With CPS, it is necessary that these devices and systems are secure and resilient. There is a need to investigate means to provide security within these systems including lightweight device handshakes and active monitoring for anomalies and intrusions. Along with security, resiliency within a IoT/CPS system is of concern. Resiliency requires redundancy of communication and ability to recover from an anomaly. Finally, the ability to maintain a resilient and secure system must remain be scalable as a system grows from 10 devices to 1000s of devices. The goal of this dissertation will be to address these issues and outline a scalable, resilient and secure architecture for IoT/CPS Systems.

The remainder of this dissertation is organized as follows: Chapter~\ref{chap:lit} is a review of literature covering CPS, security, scalability, and resiliency in IoT/CPS systems. Chapter~\ref{chap:questions} is an overview of research questions we intend to answer during this research. Finally, Chapter~\ref{chap:app} explains the approach and metrics we will use during the research process. 


\chapter{Review of Literature}
\label{chap:lit}

\subfile{chapters/related_work.tex}

\chapter{Research Questions}
\label{chap:questions}
\subfile{chapters/research_questions}{}

\chapter{Approach and Metrics}
\label{chap:app}
\subfile{chapters/approach.tex}{}

\subfile{chapters/metrics.tex}{}


{%\small
\bibliographystyle{unsrt}
\bibliographystyle{abbrv}
\bibliography{chapters/references}
}

\printnomenclature[0.5in] %used for the List of Abbreviations
     % All done with roman-numbered pages

\pagenumbering{roman}
\appendix
\titleformat{\chapter}[display]{\Large\bfseries}{Appendix \thechapter}{2pt}{\Large}
\begin{singlespace}

\chapter*{Appendices\addcontentsline{toc}{chapter}{Appendices}}
\subfile{chapters/publications-plans.tex}{}
\subfile{chapters/research_questions-scoping.tex}{}

\end{singlespace}

\end{document}